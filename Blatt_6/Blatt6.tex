\documentclass[a4paper]{scrartcl}
\usepackage[utf8]{inputenc}

\usepackage{scrpage2}
\usepackage{rotating}
\usepackage{listings}
\usepackage{amsmath}
\usepackage{amsfonts}
\usepackage{enumerate}
%\usepackage{mathtools}
\usepackage{hyperref}

\setlength{\parindent}{0mm}

\pagestyle{scrheadings}
\renewcommand{\thesubsection}{\alph{subsection}}
\clearscrheadfoot

\ohead[]{Nikolas Zeitler, Joshua Hartmann, Alexander Diegel}
\cfoot[\pagemark]{\pagemark}

\author{Nikolas Zeitler, Joshua Hartmann, Alexander Diegel}
\title{Maschinelles Lernen Blatt 5}

%INFO bitte mit TODO-tags arbeiten, dann sieht man es im Studio auf der linken seite
%TODO Beispiel 

\begin{document}
\maketitle
\section{Fragen zur Vorlesung}

\begin{enumerate}[a)]
	\item Knoten und Kanten im Bayesschen Netz:\\
	Knoten repräsentieren Wahrscheinlichkeiten von Zufallsvariablen.
	Kanten repräsentieren kausale Abhängigkeiten (conditional probabilities).
	
	\item Konditionierung: Es interessieren nur einzelne Reihen oder Spalten aus Wahrscheinlichkeitstabellen. Wird bekannt, dass eine Zufallsvariable einen bestimmten Wert annimmt, kann die Verteilung durch Marginalisierung der Tabelle nach der Ebene, bei der die Zufallsvariable den entsprechenden Wert hat, berechnet werden.
	
	\item Inferenz: Ist eine oder mehrere Variablen bekannt, ist man an der Verteilung einer/mehrerer unbekannten variablen interessiert.
	
	\item Marginalisierung integriert über Variablen, die nicht von Interesse sind. 
\end{enumerate}


\section{Bayessche Netze – Berechnung}

\begin{enumerate}[(a)]
	\item 
	\item 
	\item 
	\item 
	\item 
\end{enumerate}


\section{Bayessche Netze – Programmierung}



\section{Fragen zur Vorlesung}

\begin{enumerate}[(a)]
	\item Die Ordnung eines HMM beschreibt die Anzahl an vorherigen Zuständen, von denen der aktuelle Zustand abhängt.
	
	\item Die Matrix A enthält die Werte für $a_{ij}$. Diese geben die Übergangswahrscheinlichkeit von $\omega_i$ zu $\omega_j$ an.\\
	
	Die Matrix B enthält die Werte für $b_{jk}$. Diese geben die Wahrscheinlichkeit an, dass Zustand $\omega_j$ sich in Visible State $v_k$ äußert.\\
	
	\item Die Menge $\Omega$ beschreibt die Menge der möglichen Zustände $\omega_i$.\\
	
	Die Menge $V$ beschreibt die Menge der Visible States $v_i$.
	\item 
	\item 
	\item 
\end{enumerate}



\end{document}
